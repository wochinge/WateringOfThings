\section{Einleitung}
Nachdem JavaScript früher überwiegend bei der Entwicklung im Frontend zum Einsatz kam, profitieren JavaScript-Entwickler immer mehr vom Paradigma \textit{Learn once, write anywhere}. So schickt sich JavaScript an, mit dem Boom von NodeJS auch serverseitig verstärkt eine Rolle zu spielen. Mobile Applikationen konnten jedoch bisher nur eingeschränkt mittels JavaScript umgesetzt werden. Zu groß waren die Limitierungen durch den Fakt, dass mit JavaScript erstellte Applikationen nur innerhalb einer Web-View auf den Geräten ausgeführt werden konnten. Auf einem Hackathon entwickelte Facebook jedoch 2013 eine Möglichkeit React-Komponenten durch native UI-Komponenten auf den Geräten zu visualisieren. Dieser Prototyp wurde daraufhin innerhalb interner Projekte bei Facebook weiterentwickelt, bis eine erste Version von React Native schließlich im März 2015 als Open-Source-Projekt veröffentlicht wurde. Facebook hatte bei der Entwicklung ursprünglich lediglich die iOS-Plattform im Blick. Rasch wurde jedoch die Möglichkeit der Wiederverwendung für Android erkannt, sodass React Native seit Sptember 2015 auch für Android verfügbar ist.
Die Bereitstellung von React Native als Open-Source-Projekt sorgte von Beginn an für eine große Community, die React Native seitdem unter Facebooks Schirmherrschaft in großen Schritten vorantreibt. Über 9000 Commits von mehr als 1000 Entwicklern sowie ein zweiwöchiger-Release-Rhythmus zeugen von dem Bedarf und dem Potenzial der  Entwicklung nativer Smartphone-Applikationen mittels JavaScript.