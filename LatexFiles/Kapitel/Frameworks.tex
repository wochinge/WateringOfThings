\section{Frameworks}
React Native konzentriert sich auf einige wenige Kernfunktionen. Zur Erweiterung der Funktionalität müssen dem Projekt daher entweder geeignete Bibliotheken hinzugefügt werden oder Code selbst geschrieben werden. Insbesondere falls es für eine gewünschte plattformspezifische Funktionalität noch kein Gegenstück in React Native gibt, muss mittels nativem Code eine Bridge zwischen beidem geschrieben werden. React Native besitzt allerdings eine sehr engagierte Community, sodass dies nur selten notwendig wird. Nachfolgend wird deren Einbindung gezeigt sowie eine Auswahl nützlicher Bibliotheken vorgestellt.

\subsection{Einbindung}

\subsection{Navigation}

\subsection{Datenbank}

\subsection{Internationalisierung}

%TODO
\subsection{Testing}